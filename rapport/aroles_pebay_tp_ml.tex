\documentclass[a4paper,12pt]{article}

\usepackage[utf8]{inputenc}  % codage des caractères
\usepackage{hyphsubst} % babel râle sans ça
\usepackage[french]{babel} % pour respecter les conventions françaises (listes à puces…) et couper les mots correctement
\usepackage[T1]{fontenc} % gestion d'accents
\usepackage{setspace}
\usepackage{indentfirst}
\usepackage[skip=1em, indent]{parskip}

\usepackage{amsmath} % INDISPENSABLE ; voir si besoin documentation sur le site ams.org
\usepackage{amsfonts}  \usepackage{amssymb} % comme leur non l'indique

\sloppy % pour que les lignes ne débordent pas dans la marge !

\title{Apprentissage statistique : TP noté}  % {} laisser le champ vide
\author{Pierre Pebay et Antoine Aroles} 
\date{\today}  % {} laisser le champ vide

\begin{document}
\maketitle

\section{Introduction}
Le but de ce TP est d'entrainer un modèle de classification permettant de déterminer si une vache est dans son état normal ou non à partir du temps qu'elle a passé dans les différentes parties de l'étable au cours des dernière 24h. Pour cela, nous disposons de données recueillies en hiver dans une étable divisée en trois 


\end{document}